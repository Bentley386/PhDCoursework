\documentclass{article}
\usepackage{amsmath,amssymb,amsfonts}
\usepackage{hyperref}

\title{Ponovitev linearne algebre s primeri}
\date{Priporočljivo je, da bralec vse rezultate tudi sam preveri in jim ne zgolj slepo zaupa.}
\begin{document}
\maketitle
\section*{Preliminarne definicije}
Preslikava $T: \mathbb{R}^n \to \mathbb{R}^n$ je linearna, če za vsak $x,y \in \mathbb{R}^n, \alpha \in \mathbb{R}$ velja 
\begin{align*}
&T(x+y) = T(x)+T(y)\\
&T(\alpha x) = \alpha T(x)
\end{align*}
Z izbiro baze vektorskega prostora $\mathbb{R}^n$, lahko $T$ predstavimo z matriko $A \in \mathbb{R}^{n \times n}$.
Definirajmo še jedro (kernel) in sliko (image):
\begin{align*}
&\mathrm{ker}(A) = \{ x \in \mathbb{R}^n | Ax = 0\} \\
&\mathrm{im}(A) = \{ Ax | x \in \mathbb{R}^n\} 
\end{align*}
Dimenziji slednjega pogosto rečemo tudi rang matrike $A$. Velja še:
\begin{equation*}
\mathrm{dim}(\mathrm{ker}(A)) + \mathrm{dim}(\mathrm{im}(A)) = n
\end{equation*}
Če za $x \neq 0$ velja 
\begin{equation*}
Ax = \lambda x
\end{equation*}
pravimo, da je $x$ lastni vektor matrike $A$ z lastno vrednostjo $\lambda$.
\vspace{1em}

\noindent Če zgornjo enačbo nekoliko prepišemo dobimo $(A- \lambda I)x = 0$, kar je lahko (za $x \neq 0$) res le, če velja $\mathrm{det}(A - \lambda I) = 0$.
Iz tega dobimo postopek za izračun lastnih vrednosti matrike $A$. Računamo ničlo karakterističnega polinoma:
\begin{equation*}
p_A(\lambda) = \mathrm{det}(A-\lambda I) = 0
\end{equation*}
To je polinom stopnje $n$, in po osnovnem izreku algebre dobimo $n$ lastnih vrednosti $\lambda_i$ (štetih z večkratnostjo ničel).
Stopnji ničle $\lambda$ rečemo algebraična kratnost in označimo z $\mathrm{akr}(\lambda)$.
Lastni vektor, ki pripada lastni vrednosti $\lambda$ dobimo z reševanjem $Ax = \lambda x$ oziroma
\begin{equation*}
(A-\lambda I)x = 0,
\end{equation*}
torej lastni vektorji so elementi $\mathrm{ker}(A-\lambda I)$. Dimenziji jedra, rečemo geometrijska kratnost $\lambda$ in jo označimo z $\mathrm{gkr}(\lambda)$. V nadaljevanju bomo pogosto zlorabljali jezik in z besedo lastni vektor označevali le bazne vektorje jedra (in rekli npr. {}"Imamo dva lastna vektorja{}", čeprav jih je v resnici neskončno).

\section*{Diagonalizacija matrik in aplikacija}
Če za vsako lastno vrednost $\lambda$ velja $\mathrm{akr}(\lambda) = \mathrm{gkr}(\lambda)$ (vedno "{}imamo enako število lastnih vektorjev kot je stopnja lastne vrednosti"{}), je matrika diagonalizabilna in jo lahko zapišemo v obliki
\begin{equation*}
A = P D P^{-1},
\end{equation*}
kjer je $D$ diagonalna $n \times n$ matrika z lastnimi vrednostmi po diagonali, P pa ima pripadajoče lastne vektorje za stolpce.

\subsection*{Naloga 1}
\subsubsection*{Navodilo}
Dana je matrika 
\begin{equation*}
A = \begin{pmatrix}
-1 & -3 \\
0  & 2
\end{pmatrix}
\end{equation*}
Izračunaj $A^n$ za poljuben (naravni) $n$.
\subsubsection*{Rešitev}
Če se matriko da diagonalizirati, lahko to uporabimo za enostaven izračun poljubne potence matrike $A$. Poskusimo jo torej diagonlizirati:
\begin{equation*}
|A-\lambda I | =
\begin{vmatrix}
-1-\lambda & -3 \\
0  & 2-\lambda
\end{vmatrix} = (-1-\lambda)(2-\lambda) = 0,
\end{equation*}
iz česar takoj dobimo $\lambda_1 = -1$, $\lambda_2 = 2$. Uporabili smo znano formulo za determinanto $2 \times 2$ matrik:
\begin{equation*}
\begin{vmatrix}
a &b \\
c & d
\end{vmatrix} = ad - bc
\end{equation*}
Poiščimo lastni vektor za $\lambda = -1$:

\begin{align*}
A+I  =&
\begin{pmatrix}
0 & -3 \\
0  & 3
\end{pmatrix} \\
\begin{pmatrix}
0 & -3 \\
0  & 3
\end{pmatrix} 
\begin{pmatrix}
x \\ y \end{pmatrix} &= \begin{pmatrix} 0 \\ 0 \end{pmatrix}
\end{align*}
Iz zgornjega sistema dobimo $y=0$. Torej je element jedra oblike $\begin{pmatrix} x \\ y \end{pmatrix} = x \begin{pmatrix} 1 \\ 0 \end{pmatrix}$, kar nam da prvi lastni vektor (alternativno bi lahko ta lastni vektor tudi kar uganili).
Podobno dobimo, da je lastni vektor, ki pripada $\lambda = 2$ enak $\begin{pmatrix} -1 \\ 1 \end{pmatrix}$.
Takoj lahko torej zapišemo prehodno matriko in diagonalno matriko:
\begin{align*}
P  =&
\begin{pmatrix}
1 & -1 \\
0  & 1
\end{pmatrix} \\
D =& 
\begin{pmatrix}
-1 & 0 \\
0  & 2
\end{pmatrix} \\
P^{-1} =& 
\begin{pmatrix}
1 & 1 \\
0  & 1
\end{pmatrix}
\end{align*}
Inverz smo dobili s pomočjo formule:
\begin{equation*}
\begin{pmatrix}
a & b \\
c  & d
\end{pmatrix} ^{-1}
 = \frac{1}{ad-bc}
\begin{pmatrix}
d & -b \\
-c  & a
\end{pmatrix} 
\end{equation*}
Potence matrika $A$ lahko enostavno izračunamo s pomočjo naslednjega trika:
\begin{align*}
A =& P D P^{-1} \\
A^n =& P D^n P^{-1}
\end{align*}
Potenca diagonalne matrike je enostavno izračunljiva (potenciramo vsakega diagonalca posebej), kar da:
\begin{equation*}
A = 
\begin{pmatrix}
1 & -1 \\
0  & 1
\end{pmatrix}
\begin{pmatrix}
(-1)^n & 0 \\
0  & 2^n
\end{pmatrix}
\begin{pmatrix}
1 & 1 \\
0  & 1
\end{pmatrix}
\end{equation*}
Po množenju končno dobimo:
\begin{equation*}
A^n =
\begin{pmatrix}
(-1)^n & (-1)^n - 2^n \\
0  & 2^n
\end{pmatrix} 
\end{equation*}
Kot test, če smo se kje zmotili, lahko vstavimo $n=1$ in opazimo, da dobimo spet originalno matriko $A$.
\subsection*{Naloga 2}
\subsubsection*{Navodilo}
Dana je matrika
\begin{equation*}
A = \begin{pmatrix}
2 & 0 & 0 \\
1  & 2 & 1 \\
-1 & 0 & 1
\end{pmatrix}
\end{equation*}
Diagonaliziraj jo, oziroma pokaži, da se je ne da.
\subsubsection*{Rešitev}
\begin{equation*}
|A-\lambda I | =
\begin{vmatrix}
2-\lambda & 0 & 0 \\
1  & 2-\lambda & 1 \\
-1 & 0 & 1-\lambda
\end{vmatrix} = (2-\lambda) \begin{vmatrix} 2-\lambda & 1 \\ 0 & 1-\lambda \end{vmatrix} = (2-\lambda)^2(1-\lambda) = 0
\end{equation*}
iz česar dobimo $\lambda_{1,2} = 2, \lambda_3 = 1$.
Za izračun $3 \times 3$ determinant lahko uporabimo eksplicitno formulo (če jo znamo na pamet) ali pa razvoj po vrsticah/stolpcu. V tem primeru smo uporabili razvoj po prvi vrstici (v praksi seveda razvijemo po vrstici/stolpcu, ki ima veliko ničel). Za več informaciji o takem računanju determinante glej \href{https://en.wikipedia.org/wiki/Laplace\_expansion}{https://en.wikipedia.org/wiki/Laplace\_expansion}
\vspace{1em}

\noindent Poiščimo lastne vektorje za $\lambda = 1$:

\begin{align*}
A-I  =&
\begin{pmatrix}
1 & 0 & 0 \\
1  & 1 & 1 \\
-1 & 0 & 0
\end{pmatrix} \\
\begin{pmatrix}
1 & 0 & 0 \\
1  & 1 & 1 \\
-1 & 0 & 0
\end{pmatrix}
\begin{pmatrix}
x \\ y \\ z \end{pmatrix} &= \begin{pmatrix} 0 \\ 0 \\ 0\end{pmatrix}
\end{align*}
Iz zgornjega sistema dobimo $x=0, y=-z$. 
Torej je element jedra oblike $\begin{pmatrix} x \\ y \\ z \end{pmatrix} = y \begin{pmatrix} 0 \\ 1 \\ -1 \end{pmatrix}$, kar nam da prvi lastni vektor.
Poiščimo lastne vektorje še za $\lambda = 2$:

\begin{align*}
A-2I  =&
\begin{pmatrix}
0 & 0 & 0 \\
1  & 0 & 1 \\
-1 & 0 & -1
\end{pmatrix} \\
\begin{pmatrix}
0 & 0 & 0 \\
1  & 0 & 1 \\
-1 & 0 & -1
\end{pmatrix}
\begin{pmatrix}
x \\ y \\ z \end{pmatrix} &= \begin{pmatrix} 0 \\ 0 \\ 0\end{pmatrix}
\end{align*}
Iz zgornjega sistema dobimo $x=-z$. 
Torej je element jedra oblike $\begin{pmatrix} x \\ y \\ z \end{pmatrix} = y \begin{pmatrix} 0 \\ 1 \\ 0 \end{pmatrix} + x \begin{pmatrix}1 \\ 0 \\ -1 \end{pmatrix}$, kar nam da še dva lastna vektorja. Opazimo, da je za vsako lastno vrednost geometrijska kratnost enaka algebraični, torej se matriko $A$ da diagonalizirati:
\begin{align*}
P  =&
\begin{pmatrix}
0 & 0 & 1\\
1  & 1 & 0 \\
-1 & 0 & -1
\end{pmatrix} \\
D =& 
\begin{pmatrix}
1 & 0 & 0 \\
0  & 2 & 0 \\
0 & 0 & 2
\end{pmatrix} \\
P^{-1} =& 
\begin{pmatrix}
-1 & 0 & -1\\
1  & 1 & 1 \\
1 & 0 & 0
\end{pmatrix}
\end{align*}
Inverz smo dobili s pomočjo formule:
\begin{equation*}
A^{-1} = \frac{1}{\mathrm{det}(A)} Cof(A)^T ,
\end{equation*}
kjer smo z $Cof(A)$ označili kofaktorsko matriko za $A$: Komponenta $ij$ matrike $Cof(A)$ je $(-1)^{i+j}$ krat "{}determinanta matrike, ki ostane, ko "{}skrijemo"{} i-to vrstico in j-ti stolpec"{}. Za konkretnejšo razlago in primere glej \href{https://en.wikipedia.org/wiki/Adjugate\_matrix}{https://en.wikipedia.org/wiki/Adjugate\_matrix} 

\subsection*{Naloga 3}
\subsubsection*{Navodila}
Dana je matrika 
\begin{equation*}
A = \begin{pmatrix}
1 & 1 \\
-1  & 1
\end{pmatrix}
\end{equation*}
Izračunaj $A^n$ za poljuben (naravni) $n$.
\subsubsection*{Rešitev}
\begin{equation*}
|A-\lambda I | =
\begin{vmatrix}
1-\lambda & 1 \\
-1  & 1-\lambda
\end{vmatrix} = (-1-\lambda)^2 + 1 = 0,
\end{equation*}
Dobimo $(-1-\lambda)^2 = -1$ oziroma $\lambda_1 = 1+i, \lambda_2 = 1-i$ (lahko bi tudi kvadrirali in ničle dobili z diskriminanto). Kot vemo so lastne vrednosti realnih matrik lahko tudi kompleksne in vedno nastopajo v konjugiranih parih.
\vspace{1em}

\noindent Poiščimo lastni vektor za $\lambda = 1+i$:

\begin{align*}
A-(1+i)I  =&
\begin{pmatrix}
-i & 1 \\
-1  & -i
\end{pmatrix} \\
\begin{pmatrix}
-i & 1 \\
-1  & -i
\end{pmatrix} 
\begin{pmatrix}
x \\ y \end{pmatrix} &= \begin{pmatrix} 0 \\ 0 \end{pmatrix}
\end{align*}
Iz zgornjega sistema dobimo $x = -iy$. Torej je element jedra oblike $\begin{pmatrix} x \\ y \end{pmatrix} = y \begin{pmatrix} -i \\ 1 \end{pmatrix}$, kar nam da prvi lastni vektor. Če ga konjugiramo, dobimo lastni vektor, ki ustreza $\lambda = 1-i$.
Takoj lahko torej zapišemo prehodno matriko in diagonalno matriko:
\begin{align*}
P  =&
\begin{pmatrix}
-i & i \\
1  & 1
\end{pmatrix} \\
D =& 
\begin{pmatrix}
1+i & 0 \\
0  & 1-i
\end{pmatrix} \\
P^{-1} =& 1/2
\begin{pmatrix}
i & 1 \\
-i & 1
\end{pmatrix}
\end{align*}
Pri računanju inverza pride prav identiteta $1/i = -i$
\vspace{1em}

\noindent Izračunamo zdaj potenco po istem postopku kot prej. Da bo manj pisanja označimo zaenkrat $\lambda_1 = 1+i, \lambda_2 = 1-i$:
\begin{equation*}
A^n = 0.5
\begin{pmatrix}
-i & i \\
1  & 1
\end{pmatrix}
\begin{pmatrix}
\lambda_1^n & 0 \\
0  & \lambda_2^n
\end{pmatrix}
\begin{pmatrix}
i & 1 \\
-i  & 1
\end{pmatrix}
\end{equation*}
Ko zmnožimo dobimo:
\begin{equation*}
A^n = 0.5
\begin{pmatrix}
\lambda_1^n + \lambda_2^n & i(\lambda_2^n - \lambda_1^n) \\
i(\lambda_1^n - \lambda_2^n)  & \lambda_1^n+\lambda_2^n
\end{pmatrix}
\end{equation*}
Torej, izračunati moramo potence kompleksnega števila. Ponavadi je to veliko lažje narediti v polarnem zapisu:
\begin{align*}
&\lambda_1 = 1+i \to \lambda_1 = \sqrt{2} e^{i \pi/4} \\
&\lambda_2 = 1-i \to \lambda_2 = \sqrt{2} e^{-i \pi/4} 
\end{align*}
\begin{align*}
&\lambda_1^n = \sqrt{2}^n e^{i n \pi/4} \\
&\lambda_2^n = \sqrt{2}^n e^{-i n \pi/4} 
\end{align*}
Torej
\begin{equation*}
A^n = 0.5 \sqrt{2}^n
\begin{pmatrix}
e^{i n \pi/4} + e^{-i n \pi/4} & -i(e^{i n \pi/4} - e^{-i n \pi/4}) \\
i(e^{i n \pi/4} - e^{-i n \pi/4})  &e^{i n \pi/4} + e^{-i n \pi/4}
\end{pmatrix}
\end{equation*}
Zgornje je seveda že pravi rezultat, vendar ni v najlepši obliki. Namreč, vemo, da mora biti $A^n$ realna matrika (ker je $A$ realna), vendar ta realnost iz zgornjega zapisa ni očitna. Za interpretacijo rezultata nam je veliko bližje, če se v zgornjem zapisu uspemo znebiti vseh imaginarnih enot. To storimo z naslednjima identitetama, ki sta posledici znane formule $e^{ix} = \cos(x) + i \sin(x)$. 
\begin{align*}
&e^{ix} + e^{-ix} = 2 \cos(x) \\
&e^{ix} - e^{-ix} = 2i \sin(x)
\end{align*}
S tem lahko zapišemo končni rezultat:
\begin{equation*}
A^n = \sqrt{2}^n
\begin{pmatrix}
\cos(n \pi/4) & \sin(n \pi/4) \\
-\sin(n \pi/4)  & \cos(n \pi/4)
\end{pmatrix}
\end{equation*}
Če v zgornjo enačbo vstavimo $n=1$ se lahko prepričamo, da se najverjetneje nismo zmotili pri izračunih.

\section*{Intermezzo - Jordanova normalna forma}
Do sedaj smo imeli srečo in je matrika $A$ bila vedno diagonalizabilna. V resnici se to seveda ne zgodi vedno, ampak imamo lahko kakšno (ali več) lastno vrednost, za katero velja $\mathrm{gkr}(\lambda) < \mathrm{akr}(\lambda)$ oziroma - na voljo imamo manj lastnih vektorjev, kot je stopnja $\lambda$ v karakterističnem polinomu. V tem primeru matrike ne moremo diagonalizirati, vseeno pa jo lahko spravimo v podobno enostavno obliko - Jordanovo normalno formo:
\begin{equation*}
A = P J P^{-1}
\end{equation*}
Po stolpcih $P$ so tokrat posplošeni lastni vektorji, matrika $J$ pa je zgolj bločno diagonalna:
\begin{equation*}
J =  
\begin{pmatrix}
    J(\lambda_1) & 0 & 0 & \cdots & 0 \\
    0 & J(\lambda_2) & 0 & \cdots & 0 \\
    0 & 0 & J(\lambda_3) & \cdots & 0 \\
    \vdots & \vdots & \vdots & \ddots & \vdots \\
    0 & 0 & 0 & \cdots & J(\lambda_k)
\end{pmatrix},
\end{equation*}
kjer je $J(\lambda)$ Jordanova kletka:
\begin{equation*}
J(\lambda) = 
\begin{pmatrix}
    \lambda & 1      & 0      & \cdots & 0 \\
    0      & \lambda & 1      & \cdots & 0 \\
    0      & 0      & \lambda & \cdots & 0 \\
    \vdots & \vdots & \vdots & \ddots & 1 \\
    0      & 0      & 0      & \cdots & \lambda
\end{pmatrix}
\end{equation*}
Pri računanju lahko obravnavamo vsako lastno vrednost $\lambda$ posebej. Izračunamo pripadajoče Jordanove kletke (eni lastni vrednosti lahko pripada več kot ena kletka!) in primerne posplošene lastne vektorje in na koncu vse skupaj združimo v Jordanovo normalno formo.

Splošen postopek določanja velikosti Jordanovih kletk in iskanja posplošenih vektorjev je nekoliko zakompliciran, zato ga tu ne bomo omenjali (če koga zanima, naj pogleda kakšen textbook ali skripto). Omejili se bomo na tehnike, ki so primerne za majhne matrike. Pri vajah bomo imeli opravka zgolj z $2 \times 2$ in $3 \times 3$ matrikami, čeprav se seveda v praktičnih primerih pogosto pojavljajo večje matrike.

\noindent Najprej se bomo omejili na primer, ko velja $1 = \mathrm{gkr}(\lambda) < \mathrm{akr}(\lambda) = k$. V tem primeru se izkaže, da pripada lastni vrednosti $\lambda$ le ena Jordanova kletka in ta je velikosti $k \times k$.
Kako pa najdemo pripadajoče posplošene lastne vektorje $v_1, \dots, v_k$?
Za $v_1$  izberemo poljubni vektor iz prostora:
\begin{equation*}
v_1 = \mathrm{ker}((A-\lambda I)^k) - \mathrm{ker}((A-\lambda I)^{k-1})
\end{equation*}
Preostale pa dobimo potem rekurzivno kot:
\begin{align*}
v_2 &= (A-\lambda I) v_1 \\
v_3 &= (A-\lambda I)v_2 \\
....& \\
v_k &= (A-\lambda I) v_{k-1}
\end{align*}
in jih potem zložimo v matriko $P$ v obratnem vrstnem redu:
\begin{equation*}
P = 
\begin{pmatrix}
v_k & v_{k-1} & \dots & v_1 
\end{pmatrix}
\end{equation*}
Opomba: Lahko opazimo, da velja $v_k = (A-\lambda I) v_{k-1} = (A-\lambda I)^{k-1} v_1$. Iz tega sledi, da je 
$(A-\lambda I) v_k = (A-\lambda I)^k v_1 = 0$ oziroma $A v_k = \lambda v_k$. $v_k$ je torej lastni vektor z lastno vrednostjo $\lambda$. To lahko uporabimo kot test, da preverimo, če smo se vmes v računanju kje zmotili.

\subsection*{Naloga 4}
\subsubsection*{Navodilo}
Dana je matrika 
\begin{equation*}
A = \begin{pmatrix}
3 & 1 \\
-1  & 1
\end{pmatrix}
\end{equation*}
Izračunaj $A^n$ za poljuben (naravni) $n$.
\subsubsection*{Rešitev}
Rešitev že znamo izračunati, če je $A$ diagonalizabilna. Izkaže se, da je potence možno računati tudi s pomočjo Jordanove forme. Torej, v vsakem primeru najprej poiščemo lastne vrednosti.
\begin{equation*}
|A-\lambda I | =
\begin{vmatrix}
3-\lambda & 1 \\
-1  & 1-\lambda
\end{vmatrix} = (3-\lambda)(1-\lambda)+1 = \lambda^2 -4 \lambda + 4 = (\lambda-2)^2 = 0
\end{equation*}
Iz česar takoj dobimo $\lambda=2$, z algebraično kratnostjo ($k$ v prejšnjem poglavju) enako $2$.
Poiščimo lastne vektorje za $\lambda = 2$:
\begin{align*}
A-2I  =&
\begin{pmatrix}
1 & 1 \\
-1  & -1
\end{pmatrix} \\
\begin{pmatrix}
1 & 1 \\
-1  & -1
\end{pmatrix} 
\begin{pmatrix}
x \\ y \end{pmatrix} &= \begin{pmatrix} 0 \\ 0 \end{pmatrix}
\end{align*}
Iz zgornjega sistema dobimo $x=-y$. Torej je element jedra oblike $\begin{pmatrix} x \\ y \end{pmatrix} = x \begin{pmatrix} 1 \\ -1 \end{pmatrix}$, kar nam da prvi lastni vektor (opomba: lastne vektorje tu vedno računamo z reševanjem sistema, ker je to dokaj sistematično, ampak obstajajo pa tudi hitrejše metode. Lahko bi npr. opazili, da ima matrika rang 1 (in je torej dimenzija jedra 1), kar nam že takoj pove, da imamo samo en lastni vektor, ki ga pa potem lahko uganemo). Opazimo torej, da velja $1 =\mathrm{gkr}(2) < \mathrm{akr}(2) = 2$ - matrika ni diagonalizablna. 

\noindent Smo pa v primeru, razloženem v prejšnjem poglavju: ker je $\mathrm{gkr}(2)=1$, vemo da imamo samo eno Jordanovo kletko, ki je velikosti $2 \times 2$. Matriko $J$ v $A = P J P^{-1}$ torej pravzaprav že imamo:
\begin{equation*}
J = \begin{pmatrix}
2 & 1 \\
0 & 2
\end{pmatrix}
\end{equation*}
Za posplošene lastne vektorje sledimo receptu.
Prvega vzamemo iz $v_1 \in \mathrm{ker}((A-2I)^2) - \mathrm{ker}(A-2I)$.
Torej najprej izračunamo:
\begin{equation*}
(A-2I)^2 = \begin{pmatrix}
0 & 0 \\
0 & 0
\end{pmatrix}
\end{equation*}
Za $v_1$ vzamemo vektor, ki je v jedru te matrike (jedro te matrike je kar $\mathbb{R}^2$, saj je ničelna) in ni v jedru $(A-2I)$.
Seveda hočemo vzeti čim bolj enostavnega. Vzeli bomo kar
\begin{equation*}
v_1 = \begin{pmatrix} 1 \\ 0 \end{pmatrix}
\end{equation*}
Drugega dobimo spet po receptu iz prejšnjega poglavja:
\begin{equation*}
v_2 = \begin{pmatrix} 1 & 1 \\ -1 & -1 \end{pmatrix}
\begin{pmatrix} 1 \\ 0 \end{pmatrix} = \begin{pmatrix} 1 \\ -1 \end{pmatrix}
\end{equation*}
Vidimo, da je $v_2$ lastni vektor matrike $A$, torej se najbrž nismo zmotili.
Zapišimo še prehodno matrike (pazi: rekli smo, da dajamo vektorje v obratnem vrstnem redu v prehodno matriko!)
\begin{align*}
P  =&
\begin{pmatrix}
1 & 1 \\
-1  & 0
\end{pmatrix} \\
J =& 
\begin{pmatrix}
2 & 1 \\
0  & 2
\end{pmatrix} \\
P^{-1} =& 
\begin{pmatrix}
0 & -1 \\
1  & 1
\end{pmatrix}
\end{align*}
Za potenco uporabimo podoben trik kot prej: $A^n = P J^n P^{-1}$.
$J$ je v splošnem bločna matrika in je potenca torej samo potenca po blokih. Vedeti moramo torej le kako se potencira Jordanove kletke. Velja:
\begin{equation*}
\begin{pmatrix}
\lambda & 1 &  & \cdots & 0 \\
0 & \lambda & 1 & \cdots & 0 \\
0 & 0 & \lambda & \cdots & 0 \\
\vdots & \vdots & \vdots & \ddots & \vdots \\
0 & 0 & 0 & \cdots & \lambda
\end{pmatrix}^n
 = 
\begin{pmatrix}
\lambda^n & \binom{n}{1} \lambda^{n-1} & \binom{n}{2} \lambda^{n-2} & \cdots & \binom{n}{k-1} \lambda^{n-(k-1)} \\
0 & \lambda^n & \binom{n}{1} \lambda^{n-1} & \cdots & \binom{n}{k-2} \lambda^{n-(k-2)} \\
0 & 0 & \lambda^n & \cdots & \binom{n}{k-3} \lambda^{n-(k-3)} \\
\vdots & \vdots & \vdots & \ddots & \vdots \\
0 & 0 & 0 & \cdots & \lambda^n
\end{pmatrix}
\end{equation*}
V našem primeru torej:
\begin{equation*}
J^n = \begin{pmatrix}
2^n & n 2^{n-1} \\
0 & 2^n
\end{pmatrix}
\end{equation*}
\begin{equation*}
A^n = P J^n P^{-1} =
 \begin{pmatrix} 
1 & 1 \\
-1 & 0
\end{pmatrix}
 \begin{pmatrix} 
n 2^{n-1} & -2^n + n 2^{n-1} \\
2^n & 2^n
\end{pmatrix}
= \begin{pmatrix}
n2^{n-1} + 2^n & n 2^{n-1} \\
-n 2^{n-1} & 2^n -n2^{n-1}
\end{pmatrix}
\end{equation*}
Če vstavimo $n=1$ spet vidimo, da smo pravilno izračunali.

\subsubsection*{Hitrejši način}
Posplošene lastne vektorje bi lahko našli tudi nekoliko hitreje.
Zdaj smo nalogo rešili tako, da smo začeli z vektorjem $v_1$ iz $\mathrm{ker}((A-2I)^2) - \mathrm{ker}(A-2I)$ in potem izračunali $v_2$ kot $v_2 = (A-2I)v_1$.

\noindent
Lahko bi šli tudi v obratni smeri.
Začnemo z vektorjem $v_2$, ki je pač lastni vektor, torej $v_2 = \begin{pmatrix} 1 \\ -1 \end{pmatrix}$ in rešujemo sistem $(A-2I)v_1 = v_2$:
\begin{equation*}
\begin{pmatrix}
1 & 1 \\
-1 & -1
\end{pmatrix}
\begin{pmatrix} x \\ y \end{pmatrix}
=
\begin{pmatrix} 1 \\ -1 \end{pmatrix}
\end{equation*}
Dobimo enačbo $x+y=1$. Torej je rešitev oblike $\begin{pmatrix} 1-y \\ y \end{pmatrix}$. Vzamimo poljubno izmed teh rešitev, za naš $v_1$. Najlažje kar $y=0$ kar da spet $v_1 = \begin{pmatrix} 1 \\ 0 \end{pmatrix}$.
\noindent
Ta metoda je morda nekoliko hitrejša (ni nam treba kvadrirati matrike).

\subsection*{Naloga 5}
\subsubsection*{Navodilo}
Zapiši matriko
\begin{equation*}
A = 
\begin{pmatrix}
-2 & 1 & 3 \\
0 & -2 & 5 \\
0 & 0 & -2
\end{pmatrix}
\end{equation*}
V obliki $A = P J P^{-1}$, kjer je $J$ Jordanova forma
\subsubsection*{Rešitev}
Ker je matrika zgornje trikotnika, je determinanta enostavno samo produkt diagonalcev:
\begin{equation*}
|A-\lambda I | = (-2-\lambda)^3 = 0
\end{equation*}
Imamo $\lambda_{1,2,3} = -2$
Izračunajmo lastne vektorje:
\begin{equation*}
A+2I = 
\begin{pmatrix}
0 & 1 & 3 \\
0 & 0 & 5 \\
0 & 0 & 0
\end{pmatrix}
\end{equation*}
Podobno kot prej dobimo, da so v jedru vektorji oblike $\begin{pmatrix} x \\ y \\ z \end{pmatrix} = x \begin{pmatrix} 1 \\ 0 \\ 0 \end{pmatrix}$
Torej imamo $\mathrm{gkr}(-2) = 1$ in lahko spet uporabimo prej omenjen recept. Vemo, da bomo imeli Jordanovo kletko dimenzije $3 \times 3$.
Za vektor $v_1$ bomo potrebovali jedra $(A+2I)^3$ in $(A+2I)^2$.
\begin{equation*}
(A+2I)^2 = 
\begin{pmatrix}
0 & 0 & 5 \\
0 & 0 & 0 \\
0 & 0 & 0
\end{pmatrix}
\end{equation*}
katerega jedro razpenjata vektorja $\begin{pmatrix} 1 \\ 0 \\ 0 \end{pmatrix}$ in $\begin{pmatrix} 0 \\ 1 \\ 0 \end{pmatrix}$.
Imamo še:
\begin{equation*}
(A+2I)^3 = 
\begin{pmatrix}
0 & 0 & 0 \\
0 & 0 & 0 \\
0 & 0 & 0
\end{pmatrix}
\end{equation*}
katerega jedro je kar $\mathbb{R}^3$.
Po receptu torej za $v_1$ izberemo (neničelni) vektor iz $\mathbb{R}^3$, ki ni v $\mathrm{ker}((A+2I)^2)$. Najenostavnejša izbira je kar $v_1 = \begin{pmatrix} 0 \\ 0 \\ 1 \end{pmatrix}$.
Preostale dobimo spet po receptu:
\begin{equation*}
v_2 = 
\begin{pmatrix}
0 & 1 & 3 \\
0 & 0 & 5 \\
0 & 0 & 0
\end{pmatrix}
\begin{pmatrix} 0 \\ 0 \\ 1
\end{pmatrix} =
\begin{pmatrix} 3 \\ 5 \\ 0 \end{pmatrix}
\end{equation*}

\begin{equation*}
v_3 = 
\begin{pmatrix}
0 & 1 & 3 \\
0 & 0 & 5 \\
0 & 0 & 0
\end{pmatrix}
\begin{pmatrix} 3 \\ 5 \\ 0
\end{pmatrix} =
\begin{pmatrix} 5 \\ 0 \\ 0 \end{pmatrix}
\end{equation*}
$v_3$ je lastni vektor, kar je naš error check, da se nismo zmotili.
Lahko zapišemo:
\begin{equation*}
P = 
\begin{pmatrix}
5 & 3 & 0 \\
0 & 5 & 0 \\
0 & 0 & 1
\end{pmatrix}
\end{equation*}

\begin{equation*}
J = 
\begin{pmatrix}
-2 & 1 & 0 \\
0 & -2 & 1 \\
0 & 0 & -2
\end{pmatrix}
\end{equation*}
$P^{-1}$ pa bi lahko izračunali po tisti formuli s kofaktorsko matriko.
Poglejmo si se kako konkretno izgleda oblika $J^n$ v tem primeru:
\begin{equation*}
J^n = 
\begin{pmatrix}
(-2)^n & n (-2)^{n-1} & (-2)^{n-2} n(n-1)/2 \\
0 & (-2)^n & n (-2)^{n-1} \\
0 & 0 & (-2)^n
\end{pmatrix}
\end{equation*}
\subsubsection*{Hitrejša pot}
Poiščimo posplošene lastne vektorje še po "hitrejši poti" (brez potenc).
Začnemo z lastnim vektorjem $v_3 = \begin{pmatrix} 1 \\ 0 \\ 0 \end{pmatrix}$
in rešimo sistem $(A+2I)v_2 = v_3$:
\begin{equation*}
\begin{pmatrix}
0 & 1 & 3 \\
0 & 0 & 5 \\
0 & 0 & 0
\end{pmatrix}
\begin{pmatrix} x \\ y \\ z
\end{pmatrix} =
\begin{pmatrix} 1 \\ 0 \\ 0 \end{pmatrix}
\end{equation*}
Dobimo enačbe $z=0$, $y=1$. Torej je rešitev oblike
$\begin{pmatrix} x \\ 1 \\ 0 \end{pmatrix}$, recimo da vzamemo kar $v_2 = \begin{pmatrix} 0 \\ 1 \\ 0 \end{pmatrix}$
Zdaj pa rešimo še $(A+2I)v_1 = v_2$:
\begin{equation*}
\begin{pmatrix}
0 & 1 & 3 \\
0 & 0 & 5 \\
0 & 0 & 0
\end{pmatrix}
\begin{pmatrix} x \\ y \\ z
\end{pmatrix} =
\begin{pmatrix} 0 \\ 1 \\ 0 \end{pmatrix}
\end{equation*}
dobimo rešitev $z=1/5$, $y = -3/5$, $x$ pa je poljuben (dajmo ga kar na nič).
Dobili smo torej:
$v_3 = \begin{pmatrix} 1 \\ 0 \\ 0 \end{pmatrix}$
$v_2 = \begin{pmatrix} 0 \\ 1 \\ 0 \end{pmatrix}$
$v_1 = \begin{pmatrix} 0 \\ -3/5 \\ 1/5 \end{pmatrix}$
in torej:
\begin{equation*}
P =
\begin{pmatrix}
1 & 0 & 0 \\
0 & 1 & -3/5 \\
0 & 0 & 1/5
\end{pmatrix}
\end{equation*}
Lahko preverimo, da spet velja $A = P J P^{-1}$ (matrika $P$ ni enolična).
Izbira metode je bolj stvar preference - "{}hitrejša"{} metoda ni spet nujno bistveno hitrejša, ker potenciranje preprostih matrik ne vzame veliko časa (pa še sistemi, ki jih rešujemo pri metodi s potenciranjem so praviloma enostavnejši). Poleg tega pa imamo pri metodi s potenciranjem na voljo še naš enostaven error-check: vektor $v_k$ mora biti lastni vektor.

\noindent
Komentar:
Opisana hitrejša metoda dobro deluje samo za manjše matrike s katerimi imamo opravka tukaj. V splošnem (npr. ko $\mathrm{gkr}(\lambda) > 1$) tale način ne deluje tako gladko in je praviloma varneje delati z jedri potenc.
\subsection*{Naloga 6}
\subsubsection*{Navodilo}
Zapiši matriko
\begin{equation*}
A = 
\begin{pmatrix}
2 & 0 & 0 \\
2 & 4 & 0 \\
1 & 2 & 2
\end{pmatrix}
\end{equation*}
V obliki $A = P J P^{-1}$, kjer je $J$ Jordanova forma
\subsubsection*{Rešitev}
Ker je matrika spodnje trikotnika, je spet determinanta enostavno samo produkt diagonalcev:
\begin{equation*}
|A-\lambda I | = (2-\lambda)^2 (4-\lambda) = 0
\end{equation*}
Imamo $\lambda_{1,2} = 2, \lambda_3 = 4$.
Izračunajmo lastne vektorje, najprej za $\lambda_3=4$:
\begin{equation*}
A-4I = 
\begin{pmatrix}
-2 & 0 & 0 \\
2 & 0 & 0 \\
1 & 2 & -2
\end{pmatrix}
\end{equation*}
Dobimo enačbo $x=0, y=z$, torej je lastni vektor oblike $\begin{pmatrix} x \\ y \\ z \end{pmatrix} = y \begin{pmatrix} 0 \\ 1 \\ 1 \end{pmatrix}$.
Za to lastno vrednost je algebraična kratnost enaka geometrijski in nimamo nobenih težav (ne rabimo Jordanovih kletk).


\noindent poglejmo si še lastne vektorje za $\lambda = 2$:
\begin{equation*}
A-2I = 
\begin{pmatrix}
0 & 0 & 0 \\
2 & 2 & 0 \\
1 & 2 & 0
\end{pmatrix}
\end{equation*}
Dobimo $x=y=0$ in je torej lastni vektor enak  $\begin{pmatrix} x \\ y \\ z \end{pmatrix} = z \begin{pmatrix} 0 \\ 0 \\ 1 \end{pmatrix}$.
Torej imamo $\mathrm{gkr}(2) = 1, \mathrm{akr}(2) = 2$. Vemo, da bomo imeli eno Jordanovo kletko velikosti $2 \times 2$.
Lahko bi šli kot prej s pomočjo izračuna $\mathrm{ker} ((A-2I)^2)$, vendar glede na to, da je kletka samo $2 \times 2$, uporabimo hitrejšo metodo. Torej vzamemo
$v_2 = \begin{pmatrix} 0 \\ 0 \\ 1 \end{pmatrix}$ in računamo $(A-2I) v_1 = v_2$:
\begin{equation*}
\begin{pmatrix}
0 & 0 & 0 \\
2 & 2 & 0 \\
1 & 2 & 0
\end{pmatrix}
\begin{pmatrix} x \\ y \\ z
\end{pmatrix} =
\begin{pmatrix} 0 \\ 0 \\ 1 \end{pmatrix}
\end{equation*}
Dobimo $x=-y$ in $x+2y = 1$, kar pomeni $x=-1, y=1$, $z$ pa poljuben (dajmo ga na nič).
Torej $v_1 = \begin{pmatrix} -1 \\ 1 \\ 0 \end{pmatrix}$ 
Ko vse skupaj združimo dobimo:
\begin{equation*}
P = 
\begin{pmatrix}
0 & -1 & 0 \\
0 & 1 & 1 \\
1 & 0 & 1
\end{pmatrix}
\end{equation*}
\begin{equation*}
J = 
\begin{pmatrix}
2 & 1 & 0 \\
0 & 2 & 0 \\
0 & 0 & 4
\end{pmatrix}
\end{equation*}
$P^{-1}$ pa bi lahko izračunali po tisti formuli s kofaktorsko matriko.
Lahko se prepričamo, da res velja $A = P J P^{-1}$. Bralec naj poizkusi isto narediti še na klasičen način (z izračunom $(A-2I)^2$)
Mimogrede še omenimo, da potenco Jordanove forme v tem primeru (po blokih z uporabo prejšnjih formul) izračunamo kot:
\begin{equation*}
J^n = 
\begin{pmatrix}
2^n & n 2^{n-1} & 0 \\
0 & 2^n & 0 \\
0 & 0 & 4^n
\end{pmatrix}
\end{equation*}

\subsection*{Naloga 7}
\subsubsection*{Navodilo}
Dano imamo matriko
\begin{equation*}
A =
\begin{pmatrix}
1 & 1 & 1 \\
0 & 1 & 0 \\
0 & 0 & 1
\end{pmatrix}
\end{equation*}
Zapiši jo kot $A=P J P^{-1}$, kjer je $J$ Jordanova forma.
\subsubsection*{Rešitev}
Hitro vidimo, da imamo lastne vrednosti $\lambda_{1,2,3} = 1$.
Poglejmo si lastne vektorje:
\begin{equation*}
A-I = 
\begin{pmatrix}
0 & 1 & 1 \\
0 & 0 & 0 \\
0 & 0 & 0
\end{pmatrix}
\end{equation*}
Kar pomeni, da je element jedra oblike $\begin{pmatrix} x \\ y \\ z \end{pmatrix} = x \begin{pmatrix} 1 \\ 0 \\ 0 \end{pmatrix} + y \begin{pmatrix} 0 \\ 1 \\ -1 \end{pmatrix}$.
To je torej prvi primer, ko imamo $2 = \mathrm{gkr}(1) < \mathrm{akr}(1)$, ki ga do zdaj še nismo obravnavali in kot vidimo se lahko pojavi tudi pri $3 \times 3$ matrikah, torej ga želimo biti sposobni rešiti.

\noindent
Izkaže se, da imamo v tem primeru $2$ jordanovi kletki (splošneje nam $\mathrm{gkr}(\lambda)$ pove število Jordanovih kletk, ki pripadajo $\lambda$). Ker imamo opravka z matriko $3 \times 3$ je edina možnost, da je ena Jordanova kletka velikosti $2 \times 2$, druga pa $1 \times1$.

\noindent
V splošnem je to lahko zoprno, za $3 \times 3$ matrike, kjer je $2 = \mathrm{gkr}(\lambda) < \mathrm{akr}(\lambda) = 3$ pa lahko povemo sledeči recept:

\noindent
Za $v_1$ vzamemo vektor iz $\mathbb{R}^3$, ki ni v $\mathrm{ker}(A-\lambda I)$. $v_2$ izračunamo kot $v_2 = (A-\lambda I) v_1$. Izkaže se, da velja $v_2 \in \mathrm{ker}(A-\lambda I)$ (pozor: ni nujno, da je $v_2$ kakšen izmed naših izbranih lastnih vektorjev - lahko je njuna linearna kombinacija). S tem dobimo vektorja, ki ustrezata $2 \times 2$ kletki.
Za tretji vektor, ki ustreza $1 \times 1$ kletki vzamemo tistega, ki je v jedru $A-\lambda I$ in je linearno neodvisen od $v_1$ in $v_2$.

\noindent
Vrnimo se torej na nalogo. Vzamemo lahko npr $v_1 = \begin{pmatrix} 0 \\ 1 \\ 0 \end{pmatrix}$.
Izračunamo:
\begin{equation*}
v_2 = (A-I)v_1 = \begin{pmatrix} 1 \\ 0 \\ 0 \end{pmatrix}
\end{equation*}
ki je res v jedru $A-I$.
Za $v_3$ vzamemo poljubnega, ki je od prvih dveh linearno neodvisen. Npr. $v_3 = \begin{pmatrix} 0 \\ 1 \\ -1 \end{pmatrix}$.

Dobimo: (Pozor, kot prej moramo za 2x2 kletko zamenjati vrstni red, ko jih pišemo v $P$):
\begin{equation*}
P = 
\begin{pmatrix}
1 & 0 & 0 \\
0 & 1 & 1 \\
0 & 0 & -1
\end{pmatrix}
\end{equation*}
\begin{equation*}
J = 
\begin{pmatrix}
1 & 1 & 0 \\
0 & 1 & 0 \\
0 & 0 & 1
\end{pmatrix}
\end{equation*}

\begin{equation*}
P^{-1} = 
\begin{pmatrix}
1 & 0 & 0 \\
0 & 1 & 1 \\
0 & 0 & -1
\end{pmatrix}
\end{equation*}
Lahko se prepričamo, da res velja $A = P J P^{-1}$.

\subsection*{Dodatek - inverz 3x3 matrike}
Kot obljubljeno, bomo tu v dodatku podali nekaj eksplicitnih izrazov za $3 \times 3$ matrike.
Recimo, da imamo matriko
\begin{equation*}
A = \begin{pmatrix} 
a & b & c \\ 
d & e & f \\ 
g & h & i 
\end{pmatrix}
\end{equation*}
Rekli smo že, da velja 
\begin{equation*}
A^{-1} = \frac{1}{\mathrm{det}(A)} Cof(A)^T
\end{equation*}
Eksplicitno, je kofaktorska matrika v tem primeru enaka:
\begin{equation*}
\mathrm{Cof}(A) = \begin{pmatrix}
\det \begin{pmatrix} e & f \\ h & i \end{pmatrix} & -\det \begin{pmatrix} d & f \\ g & i \end{pmatrix} & \det \begin{pmatrix} d & e \\ g & h \end{pmatrix} \\
-\det \begin{pmatrix} b & c \\ h & i \end{pmatrix} & \det \begin{pmatrix} a & c \\ g & i \end{pmatrix} & -\det \begin{pmatrix} a & b \\ g & h \end{pmatrix} \\
\det \begin{pmatrix} b & c \\ e & f \end{pmatrix} & -\det \begin{pmatrix} a & c \\ d & f \end{pmatrix} & \det \begin{pmatrix} a & b \\ d & e \end{pmatrix}
\end{pmatrix}
\end{equation*}
Za $A^{-1}$ torej samo zgornjo transponiramo in delimo z $\det A$. Seveda se je nesmiselno učiti zgornje formule na pamet - raje se naučimo splošen recept (prej omenjeno skrivanje vrstic/stolpcev itd.)

\noindent
Na tem mestu morda samo še omenimo eksplicitno formulo za izračun $\det A$:
\begin{equation*}
\det(A) = a(ei - fh) - b(di - fg) + c(dh - eg)
\end{equation*}
To si je tudi nesmiselno zapominti na pamet. Lahko si zapomnemo kakšne mnemonike za izračun $3 \times 3$ detreminante (najbrž smo že vsi videli tisti trik z risanjem diagonal po matriki - \href{https://www.youtube.com/watch?v=HRiANm61i0s}{https://www.youtube.com/watch?v=HRiANm61i0s}), ali pa si enostavno zapomnemo formulo za razvoj determinante po vrsticah/stolpcih.

\subsection*{Dodatek 2 - Primer iz vaj}
\subsubsection*{Navodilo}
Reši sistem diferenčnih enačb:
\begin{equation*}
y_{n+1} = A y_n + f_n,
\end{equation*}
kjer je 
\begin{equation*}
A =
\begin{pmatrix}
1 & 1 \\
-1 & 1 
\end{pmatrix}
\end{equation*}
\begin{equation*}
f_n =
\begin{pmatrix}
1  \\
0  
\end{pmatrix}
\end{equation*}
\begin{equation*}
y_0 =
\begin{pmatrix}
1  \\
-1  
\end{pmatrix}
\end{equation*}
\subsubsection*{Rešitev}
Iz teorije vemo, da nam rešitev da formula:
\begin{equation*}
y_n = A^n y_0 + \sum_{k=0}^{n-1} A^k f_{n-1-k}
\end{equation*}
Torej bomo najprej rabili potence matrike $A$. To smo izračunali v eni izmed prejšnjih nalog:
\begin{equation*}
A^n = 2^{n/2}
\begin{pmatrix}
\cos(n \pi/4) & \sin(n \pi/4) \\
-\sin(n \pi/4)  & \cos(n \pi/4)
\end{pmatrix}
\end{equation*}

Ko vstavimo to v enačbo dobimo:
\begin{equation*}
y_n = 
2^{n/2}
\begin{pmatrix}
\cos(n \pi/4) - \sin(n\pi/4) \\
-\sin(n \pi /4) - \cos(n \pi / 4)
\end{pmatrix} + \sum_{k=0}^{n-1}
2^{k/2}
\begin{pmatrix}
\cos(k \pi/4) \\
-\sin(k \pi /4)
\end{pmatrix} 
\end{equation*}

Vsota na desni se da napisati na lepši način. Zaradi preglednosti jo izračunamo po komponentah. Najprej, s pomočjo $e^{ix} = \cos(x) + i \sin(x)$ napišemo:
\begin{equation*}
\sum_{k=0}^{n-1} 2^{k/2} \cos(k \pi /4) = \mathrm{Re} \left( \sum_{k=0}^{n-1} \sqrt{2}^k e^{i k \pi/4} \right) = \mathrm{Re} \left( \sum_{k=0}^{n-1} (\sqrt{2} e^{i \pi/4})^k \right)
\end{equation*}
S pomočjo formule $\sum_{k=0}^{n-1} q^k = \frac{1-q^n}{1-q}$ lahko izrazimo zgornjo vsoto kot:
\begin{equation*}
\mathrm{Re} \left( \frac{1-2^{n/2} e^{i n \pi/ 4}}{1-\sqrt{2} e^{i \pi /4}} \right) = \mathrm{Re} \left( \frac{(1-2^{n/2} e^{i n \pi /4})(1-\sqrt{2} e^{-i\pi /4})}{(1-\sqrt{2} e^{i \pi /4})(1 - \sqrt{2} e^{-i \pi/ 4})} \right),
\end{equation*}
kjer smo kot ponavadi, ko imamo opravka s kompleksnimi ulomki, imenovalec in števec pomnožili z konjugiranim imenovalcem.
Zmnožimo faktorja:
\begin{equation*}
= \mathrm{Re} \left( \frac{1-2^{n/2} e^{i n \pi /4} - \sqrt{2} e^{-i \pi /4} + 2^{(n+1)/2} e^{i(n-1) \pi/ 4}}{1-\sqrt{2}(e^{i \pi /4} + e^{-i \pi 4}) + 2} \right).
\end{equation*}
S pomočjo identitete $e^{ix}+e^{-ix} = 2 \cos(x)$ vidimo, da je imenovalec enak $1-2\sqrt{2} \sqrt{2}/2 +2 = 1$ in ostane:
\begin{equation*}
= \mathrm{Re} \left( 1-2^{n/2} e^{i n \pi /4} - \sqrt{2} e^{-i \pi /4} + 2^{(n+1)/2} e^{i(n-1) \pi/ 4} \right)
\end{equation*}
Končno nimamo več ulomka in lahko vzamemo realni del:
\begin{equation*}
= 1-2^{n/2} \cos(n \pi / 4) - \sqrt{2} \cos(-\pi/4) + 2^{(n+1)/2}\cos((n-1) \pi/ 4)
\end{equation*}
Vidimo, da se prvi in tretji člen pokrajšata, za zadnjega pa lahko uporabimo adicijski izrek $\cos(a-b) = \cos(a)\cos(b) + \sin(a)\sin(b)$:

\noindent
OPOZORILO: Tu je bila na vajah napaka. Ob strani smo sicer napisali pravilni adicijski izrek, ampak potem uporabili napačnega: $\cos(a-b) = \cos(a)\cos(-b) + \sin(a) \sin(-b)$, kar je privedlo do napačnega rezultata.

\noindent
Če nadaljujemo po pravilni poti dobimo:
\begin{equation*}
= -2^{n/2} \cos(n \pi / 4) + 2^{(n+1)/2} \left( \cos(n \pi/4) \sqrt{2}/2 + \sin(n \pi/4) \sqrt{2}/2 \right) 
\end{equation*}
\begin{equation*}
=  2^{n/2} \sin(n \pi/4)  
\end{equation*}

Izračunati moramo še vsoto:
\begin{equation*}
\sum_{k=0}^{n-1} 2^{k/2}  \sin(k \pi / 4) = \mathrm{Im} \left( \sum_{k=0}^{n-1} (\sqrt{2} e^{i \pi /4 })^k \right), 
\end{equation*}
kar je pravzaprav ista vsota kot prej le, da vzamemo na koncu imaginarni del, torej:
\begin{equation*}
= \mathrm{Im} \left( 1-2^{n/2} e^{i n \pi /4} - \sqrt{2} e^{-i \pi /4} + 2^{(n+1)/2} e^{i(n-1) \pi/ 4} \right)
\end{equation*}
\begin{equation*}
= -2^{n/2}\sin(n \pi/4) - \sqrt{2} \sin(-\pi/4) + 2^{(n+1)/2} \sin((n-1)\pi/4)
\end{equation*}
Tokrat uporabimo adicijski izrek $\sin(a-b) = \sin(a)\cos(b) - \cos(a) \sin(b)$:
\begin{equation*}
= -2^{n/2}\sin(n \pi/4)  +1 + 2^{(n+1)/2} \left(\sin(n \pi/4) \sqrt{2}/2 - \cos(n \pi/4) \sqrt{2}/2 \right)
\end{equation*}
\begin{equation*}
= -2^{n/2}\sin(n \pi/4)  +1 + 2^{n/2} \sin(n \pi/4)  - 2^{n/2} \cos(n \pi / 4)
\end{equation*}
\begin{equation*}
= 1 - 2^{n/2} \cos(n \pi / 4)
\end{equation*}
Če se vrnemo na našo rešitev imamo torej:
\begin{equation*}
y_n =
\begin{pmatrix}
2^{n/2} ( \cos(n \pi/4) - \sin(n \pi/4)) + 2^{n/2} \sin(n \pi/4) \\
2^{n/2} (-\sin(n \pi/4) - \cos(n \pi/4))  -1  + 2^{n/2} \cos(n \pi/4)
\end{pmatrix}
\end{equation*}

\begin{equation*}
y_n =
\begin{pmatrix}
2^{n/2} \cos(n \pi/4)  \\
-1 - 2^{n/2} \sin(n \pi /4)
\end{pmatrix}
\end{equation*}
Če vstavimo še $n=0$, vidimo da to tudi res zadošča našemu začetnem pogoju.













\end{document}
