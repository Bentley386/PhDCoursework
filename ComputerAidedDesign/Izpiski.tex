\documentclass{article}

\usepackage{amsmath,amssymb,amsfonts}
\usepackage{graphicx}
\usepackage{float}
\usepackage{geometry}
\usepackage{subcaption}

\usepackage{wrapfig}

\begin{document}

\title{CAD - izpiski}
\maketitle

\section{CAD Osnove}
\subsection{Bernsteinovi Polinomi}

Za lažje razumevanje nadaljnega gradiva najprej obdelamo Bernsteinove polinome.

Dobro vemo, da so iz stališča aproksimacije polinomi pomembni, saj lahko z njimi aproksimiramo poljubno zvezno funkcijo (Weierstrassov izrek). V praktičnih primerih moramo pri delu s polinomi izbrati bazo. Izkaže se, da najbolj naivna izbira - monomi ni vedno najboljša. 

Boljša izbira so lahko Bernsteinovi polinomi - $k$-ti bernsteinov bazni polinom stopnje $p$ je podan kot 
\begin{equation}
b_k^p (t) = \binom{p}{k} (1-t)^{p-k} t^k
\end{equation}

Polinomu, zapisanem v tej bazi rečemo, da je zapisan v "Bernsteinovi obliki". 
Naštejmo nekaj dobrih lastnosti Bernsteinovih baznih polinomov:

\begin{itemize}
\item Nenegativnost - $b_k^p(t) \geq 0$.
\item Particija enote - $\sum_{k=0}^p b_k^p (t) = 1$ (sledi iz binomskega izreka).
\item Stabilnost - ničle polinoma izraženega v Bernsteinovi bazi so najmanj občutljive (v primerjavi s kako drugo bazo) na majhne spremembe koeficientov v razvoju.
\item Za določen integral, odvode in višanje reda $b_k^p(t)$ obstajajo enostavne formule.
\item S pomočjo De Casteljau-jevega algoritma lahko stabilno izračunamo vrednost polinoma v Bernsteinovi obliki.
\end{itemize}

V računalniški grafiki s pomočjo Bernsteinovih polinomov definiramo Bézierjeve krivulje:
\begin{equation}
C(t) = \sum_{k=0}^p \textbf{P}_k b_k^p (t)
\end{equation}
Vektorjem $\textbf{P}_k$ rečemo kontrolne točke in tvorijo kontrolni poligon. S spremembo pozicij kontrolnih točk lahko na interaktiven način spreminjamo obliko Bézierjeve krivulje. Lastnost, ki je nismo omenili je, da ima Bernsteinov polinom $b_k^p$ maksimum pri $t=k/p$. Spremembe $k$-te kontrolne točke se torej najbolj poznajo v bližini tega maksimuma, vendar vseeno vplivajo na celotno krivuljo.

Dobre lastnosti Bézierjevih krivulj sledijo iz lastnosti Bernsteinovih polinomov:
\begin{itemize}
\item Enostavne formule za računanje odvodov, integralov in višanja reda krivulje.
\item De Casteljau-jev algoritem za stabilno evaluacijo.
\item Afina invarianca - namesto na sami krivulji lahko, ekvivalentno, poljubno afino transformacijo apliciramo kar na kontrolni poligon (sledi iz particije enote)
\item Bézierjeva krivulja leži znotraj konveksne ogrinjače kontrolnega poligona (Sledi iz pozitivnosti in particije enote).
\item Variation diminishing - Poljubna premica nima več presečišč s krivuljo kot z kontrolnim poligonom. Geometrijsko to pomeni, da nimamo "nepotrebnih" oscilacij.
\end{itemize}

Posplošitev na ploskve je enostavna. Bézierjovo ploskev dobimo s pomočjo tenzorskega produkta Bernsteinovih polinomov:
$S(u, v) = \sum_{i=1}^n \sum_{j=1}^m \textbf{B}_{i,j} b_i^p (u) b_j^q (v)$
Pri tem se ohranijo omenjene lepe lastnosti Bézierovih krivulj, hkrati pa velja, da je rob take ploskve Bézierjeva krivulja. 


\subsection{B Zlepki}

Prej smo govorili o polinomskih prostorih. Veliko bolj bogati pa so prostori polinomskih zlepkov - ti vsebujejo funkcije izbranega reda gladkosti, ki so zožene na primerne intervale polinomi. Polinomi sami so seveda samo poseben primer polinomskih zlepkov.

B Zlepki so baza tega prostora, ki ima lepe lastnosti, podobno kot so Bernsteinovi polinomi "lepa" baza za navadne polinome.

V njihovi definicija igra pomembno vlogo seznam vozlov: $\Xi = ( \xi_1, \dots \xi_{n+p+1} )$, pri čemer se lahko kakšna vrednost vozla pojavi več kot enkrat (privzamimo še, da je seznam urejen nepadajoče). $p$ je red polinoma zlepka in $n$ število baznih funkcij za B-zlepke.

Zožen na interval med sosednjima, različnima vozloma je zlepek gladek (polinom $p$-te stopnje). Če vozel ni ponovljen, ima ob prehodu čezenj zlepek red gladkosti $C^{p-1}$. Splošneje, če je vozel ponovljen $m$ - krat, ima zlepek čezenj red gladkosti $C^{p-m}$. 

Ponavadi imamo opravka z odprtimi vozli - prvih in zadnjih $p+1$ vozlov ima isto vrednost. To povzroči, da so zlepki na robu interpolirajoči.

$k$-to bazno funkcijo stopnje $p$ označimo z $N_{k,p}(\xi)$. $N_{k,0}$ imajo enostavno obliko ($1$ na med sosednjima vozloma in $0$ sicer), višje stopnje pa dobimo rekurzivno s tako imenovano Cox-de Boor formulo.

Lastnosti $N_{k,p}$ so sledeče:
\begin{itemize}
\item Particija enote $\sum_{k=1}^n N_{k,p}(\xi) = 1$.
\item nenegativnost $N_{k,p} \geq 0$.
\item Obstajajo eksplicitne formule za odvod/integral/višanje reda/dodajanje vozla.
\item Zlepek lahko numerično stabilno izračunamo z de Boor-ovim algoritmom.
\item Nosilec za bazno funkcijo B-zlepka reda $p$ je $p+1$ elementov: $\mathrm{supp} N_{k,p} \subset [\xi_k,\xi_{k+p+1}]$.
\item Ta baza je optimalna za primeren prostor zlepkov v podobnem smislu kot so Bernsteini optimalni za polinome.
\end{itemize}


Analogno z Bézierjovimi krivuljami lahko s pomočjo teh zlepkov tvorimo krivuljo $C(\xi)$. Koeficientom pred posameznimi baznimi B-zlepki spet rečemo kontrolne točke $\textbf{P}_k$.
\begin{equation}
 C(\xi) = \sum_{k=1}^n \textbf{P}_k N_{k,p}(\xi)
\end{equation}
Pridobljena krivulja interpolira le robne točke (pri odprtih vozlih), ter morebitne točke kjer je kratnost vozla enaka $p$). 

Lastnosti te krivulje so spet podobne kot pri Bezierovih krivuljah:
\begin{itemize}
\item Afine transformacije lahko namesto na krivulji uporabimo na kontrolnem poligonu.
\item Krivulja leži v konveksni ogrinjači kontrolnih točk.
\item Kot omenjeno krivulja interpolira na robu. Tangenta na vsakem robu je nosilka daljice, ki jo tvorita robni dve kontrolni točki.
\item Variation diminishing.
\item Zaradi lokalnega nosilca, sprememba kontrolne točke spremeni krivulje le v manjši okolici (za razliko od Bezierjevih krivulj).
\end{itemize}

\subsection{NURBS}

V tem delu omenimo še trenutno enega najbolj razširjenih orodij v računalniški grafiki - NURBS (Non Uniform Rational B-Spline).
B-Zlepki (in torej tudi Bézierjeve krivulje) so posebni primeri NURBS-ov, poleg njih pa lahko izražamo še druge oblike, npr. stožnice.

Definiramo jih s pomočjo prej omenjenega seznama vozlov $\Xi$ in dodatnim seznamom uteži $W = (w_1, \dots , w_n)$.

Bazne funkcije NURBS definiramo kot:
\begin{equation}
R_{k,p} (\xi) = \frac{w_k N_{k,p} (\xi)}{\sum_{j=1}^n w_j N_{j,p}(\xi)},
\end{equation}
krivulje pa potem spet dobimo kot linearno kombinacijo takih baznih funkcij, kjer koeficientom rečemo kontrolne točke. Podobno kot prej lahko s pomočjo tenzorskega produkta dobimo tudi NURBS ploskve ali še višje dimenzionalne objekte.

Geometrijsko lahko NURBS krivuljo v $\mathbb{R}^d$ dobimo tako, da krivuljo iz B-zlepkov v $\mathbb{R}^{d+1}$ projeciramo na hiperravnino $x_{d+1}=1$.

TODO - mogoče tle še enkrat ponovit dobre lastnosti...
\section{Algoritmi / konstrukcije}
\subsection{Coonsova krpa}

Za začetek si poglejmo naslednji problem:
Imamo štiri parametrične krivulje $C_i(t)$, $i=1,2,3,4$. 
Želimo imeti parametrizirano ploskev $x(u,v)$, katerega robovi ustrezajo tem krivuljam. Torej želimo "zapolniti" notranjost krivulj. (Seveda to ni možno za poljubne $C_i (t)$, vendar se s tem problemom ne bomo obremenjevali). 
Npr želimo $x(u,0) = C_1(u)$, $x(u,1) = C_2(u)$. Lahko se omejimo na primer, ko so vse robne krivulje parametrizriane na intervalu $t \in [0,1]$.

Rešitev nam poda Coonsova krpa in se glasi:
\begin{align}
x(u,v) =& (1-u) x(0,v) + u x(1,v) + (1-v) x(u,0) + v x(u,1) - \nonumber \\&- 
\begin{pmatrix} 1-u & u \end{pmatrix} \begin{pmatrix} x(0,0) & x(0,1) \\ x(1,0) & x(1,1) \end{pmatrix}
\begin{pmatrix} 1-v \\ v \end{pmatrix},
\end{align}
kjer smo zaradi enostavnosti uporabili isto oznako $x$ za ploskev (leva stran enačbe) in robne krivulje (v členih na desni strani enačbe). Iz zgornje enačbe lahko preverimo, da $x(u,v)$ res interpolira robne krivulje.

Kot smo že videli v računalniški grafiki krivulje/ploskve opišemo s pomočjo kontrolnih točk. Ploskev opišemo s kontrolnimi točkami $\textbf{P}_{i,j}, i= 0 \dots m, j= 0 \dots n$.
Prejšnjo formulo lahko napišemo tudi na nivoju kontrolnih točk. Tako lahko definiramo diskretno Coonsovo krpo kot
\begin{align}
\textbf{P}_{i,j} =& (1-i/m) \textbf{P}_{0,j} + i/m \textbf{P}_{m,j} + (1-j/n) \textbf{P}_{i,0} + j/n \textbf{P}_{i,n} - \nonumber \\ &-
\begin{pmatrix} 1-i/m & i/m \end{pmatrix} \begin{pmatrix} \textbf{P}_{0,0} & \textbf{P}_{0,n} \\ \textbf{P}_{m,0} & \textbf{P}_{m,n} \end{pmatrix}
\begin{pmatrix} 1-j/n \\ j/n \end{pmatrix},
\end{align}
pri čemer so robne kontrolne točke znane, saj so robne krivulje znane.

Težava se lahko pojavi, če hočemo s Coonsovo krpo zapolniti krivulje, ki niso planarne.
Izkaže se namreč, da je Coonsova krpa rešitev določenega optimizacijskega problema in sicer
minimizira $\int_U x_{uv}^2(u,v) \textup{d}S$, pri pogoju, da $x(u,v)$ intrepolira robne krivulje.
Geometrijska interpretacija zgornjega funkcionala je, da Coonsova krpa minimizira "twist" ploskve.

Pri računalniški grafiki je ta minimizacija morda nezaželena (želimo, da geometrija ploskve na nek način sledi geometrijam robnih krivulj, ki so morda precej ukrivljene). Očesu bolj prijazna interpolacija je lahko takšna, ki ne minimizira twista.

Rešitev tega problema je sledeča opazka:

Recimo, da imamo dano Coonsovo krpo $x(u,v), (u,v) \in [0,1]^2$. Vzamemo dve različni točki $(u_i, v_i), i=1,2$. Ti določata pravokotnik v $[0,1]^2$. Štiri krivulje, ki tvorijo rob tega pravokotnika se preslikajo s preslikavo $x(u,v)$ preslikajo na 4 krivulje v notranjosti Coonsove krpe. Izkaže se, da če vzamemo te štiri krivuje in iz njih tvorimo novo Coonsovo krpo, dobimo spet prvotno Coonsovo krpo $x(u,v)$, zoženo na primerno podmnožico $[0,1]^2$. Temu se reče princip permanence.

Princip permanence lahko izkoristimo na naslednji način. Recimo, da imamo 3x3 mrežo kontrolnih točk (od $\textbf{P}_{i-1,j-1}$ do $\textbf{P}_{i+1,j+1}$).
Ker vse razen $\textbf{P}_{i,j}$ tvorijo rob pravokotnika, lahko $\textbf{P}_{i,j}$ z njimi izrazimo (saj vemo da mora biti Coonsova krpa).

Izraža se kot:
\begin{equation}
\textbf{P}_{i,j} = -0.25 (\textbf{P}_{i-1,j+1} + \textbf{P}_{i+1,j+1} + \textbf{P}_{i-1,j-1} + \textbf{P}_{i+1,j-1}) + 0.5 (\textbf{P}_{i,j+1} + \textbf{P}_{i-1,j} + \textbf{P}_{i+1,j} + \textbf{P}_{i,j-1})
\label{permanence}
\end{equation}

Alternativno lahko torej, diskretno Coonsovo krpo pridobimo, če zgornjo enačbo zapišemo za vsako notranjo kontrolno točko in rešimo pridobljen sistem enačb.

Kompaktneje lahko enačbo (\ref{permanence}) shematsko predstavimo z masko:
\begin{equation}
\textbf{P}_{i,j} = 0.25 \begin{pmatrix} -1  & 2 & -1 \\ 2 & * & 2 \\ -1 & 2 & -1\end{pmatrix}
\end{equation}

Posplošitev Coonsove krpe bomo dobili, če izberemo drugo masko podobne oblike:
\begin{equation}
\textbf{P}_{i,j} = \begin{pmatrix} \alpha  & \beta & \alpha \\ \beta & * & \beta \\ \alpha & \beta & \alpha\end{pmatrix}
\end{equation}

Za Coonsovo krpo velja $\alpha = -1/4, \beta = 1/2$, s spreminjanjem teh števil pa dobimo nekoliko drugačne ploskve, ki morda lepše zgledajo. 

TODO povsod (primeri/slike iz člankov?)
%
%S takimi zlepki bomo opisalo geometrijo domene, na kateri nameravamo reševat PDE.
%\section{Linear Problems}
%
%Ideja - baza za opis geometrije ista kot baza za opis rešitve.
%Rešujemo PDE na domeni $\Omega$. Ta domena je predstavljena z B-zlepki, torej imamo geometrijsko preslikavo $x : \hat{\Omega} \to \Omega$, kjer je $\hat{\Omega}$ parameter space.
%
%Na parameter space-u lahko definiramo funkcijo $\hat{u}(\xi) = \sum_i d_i N_{i,p}(\xi)$ (in podobno v več dimenzijah, vsota gre čez vse bazne funkcije), kjer so koeficienti tokrat neznani - kontrolne spremenljivke.
%Na prostoru $\Omega$ lahko definiramo funkcijo s pullbackom z inverzom geometrijske preslikave.
%
%
%Bubnov-Galerkin:
%
%
%PDE predstavimo v weak obliki. "Trial functions" (baza rešitve) izberemo iz Soboljevega prostora z danimi Dirichletovimi robnimi pogoji, "test funkcije" pa so iz Soboljevega prostora z ničelnimi Dirichletovimi pogoji.
%
%V Galerkinovi metodi te funkcije aproksimiramo s končnimi prostori. 
%Delali bomo le s prostorom ki ima ničelne Dirichletove pogoje, rešitev pa potem "liftamo" (prištejemo primerno funkcijo $g$ da zadostimo robnemu pogoju).
%
%Če našo rešitev zapišemo kot lin. komb. B-zlepkov, tistih B-zlepkov, ki so na Dirichletovem robu neničelni avtomatsko ni v razvoju.
%Po drugi strani pa ima lifting funkcija $g$ neničelne le preostale koeficiente.
%S tem pretvorimo šibko formulacijo v sistem linearnih enačb. 
%Matriko zgradimo tako da se iteriramo po elementih (lahko bi se po funkcijah ampak je portratno ker imajo majhen support).
%Integracijo delamo z kvadraturnimi formulami
%Iz weak formulacije dobimo matrični problem za koeficiente. Matrike so sparse zaradi lokalnosti B-zlepkov.
%
%
%V kolokaciji rešujemo strong form in zahtevamo da je enačbi zadoščeno na danih kolokacijskih točkah.
%Spet izrazimo rešitev kot lin. komb zlepkov z ničelnimi pogoji + lifting.
%Apliciramo diferencialni operator na rešitev (aplicira se na zlepke, formule za odvod zlepkov pa poznamo) in tako direktno sestavimo matrični sistem za neznane koeficiente.

\end{document}
