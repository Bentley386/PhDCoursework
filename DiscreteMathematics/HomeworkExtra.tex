\documentclass{article}


\usepackage[nottoc,numbib]{tocbibind}
\usepackage[slovene]{babel} 
\usepackage[utf8]{inputenc}
\usepackage{graphicx}
\usepackage{subcaption}
\usepackage{geometry}
\usepackage{amssymb}
\usepackage{caption}
\usepackage{float}
\usepackage{wrapfig}
\usepackage{romannum}
\usepackage{physics}
\usepackage{amsmath,amsfonts,amsthm,bm} % Math packages
\geometry{margin=1in}


\begin{document}
\pagenumbering{gobble}
\begin{titlepage}
    \begin{center}
        \vspace*{1cm}
        \Large
\includegraphics[width=.8\linewidth]{slike/fmflogo.pdf}\\
        \Large
\vspace{1cm}
        Permutations and combinatorial structures\\
        \huge
        \textbf{Homework 1 (Extra)\\}
\Large  
        \vspace{1cm}
        \textbf{Andrej Kolar - Po{\v z}un\\}


\vfill
\normalsize
\end{center}. 
\end{titlepage}
\newpage
\pagenumbering{arabic}
\section*{Problem statement}
Let $G$ act transitively on $\Omega$ and $\Delta$. 
Prove that the two actions are isomorphic, if and only if, there exists an $\alpha \in \mathrm{Aut}(G)$, such that $G_\omega = (G_\delta)^\alpha$ for some $\omega \in \Omega$, $\delta \in \Delta$. 
\section*{Left to right implication}
Let's assume we have an isomorphism of actions $(\alpha, \phi)$, where $\alpha \in \mathrm{Aut}(G)$ and $\phi: \Omega \to \Delta$.
By the definition of an action isomorphism we know that for each $g \in G$ and $\omega \in \Omega$ we have:
\begin{equation*}
\phi(\omega^g) = \phi(\omega)^{\alpha(g)},
\end{equation*}
where, as usual, $\omega^g$ refers to the action on $\Omega$, while the RHS refers to the action on $\Delta$.
The claim follows directly if we fix $\omega \in \Omega$ and pick $\delta = \phi(\omega)$. In that case we have:
\begin{equation*}
\phi(\omega^g) = \delta^{\alpha(g)}, \forall g \in G.
\end{equation*}
Now if $g \in G_\omega$, using the fact that $\phi, \alpha$ are both bijective, we have $\delta = \delta^{id} = \phi(\omega) = \phi(\omega^g) = \delta^{\alpha(g)}$ and therefore $\alpha(g) \in G_\delta$. We have shown $(G_\omega)^\alpha \subset G_\delta$. 
Similarly if we now take $g \in G_\delta$, we have $\phi(\omega) = \delta = \delta^g = \delta^{\alpha (\alpha^{-1}(g))} = \phi(\omega^{(\alpha^{-1}) g})$ and hence $\alpha^{-1} (g) \in G_\omega$ or $g \in \alpha(G_\omega)$. Hence $G_\delta \subset (G_\omega)^\alpha$. Therefore we have $G_\delta = (G_\omega)^\alpha$ and $(G_\delta)^{\alpha^{-1}} = G_\omega$. The claim we were trying to prove then follows by making the substitution $\alpha \to \alpha^{-1}$.
\section*{Right to left implication}
As the actions are transitive, we know they are equivalent to actions on the coset space by right multiplication (Theorem 2.1 in the lecture notes). Let us therefore assume $\Omega = G/G_\omega$ and $\Delta = G/G_\delta$, with $G$ acting on each of the two by right multiplication. Proving that these two actions are isomorphic will prove the original claim as isomorphism of actions is an equivalence relation.


In order to prove that the two actions are isomorphic, we need to find a bijective map $\phi: G/G_\omega \to G/G_\delta$ satisfying certain conditions.

Denote $\beta = \alpha^{-1} \in \mathrm{Aug}(G)$. Define $\phi(G_\omega x) = G_\delta x^{\beta}$.
This is well defined: If we have $G_\omega x_1 = G_\omega x_2$, we need to check if also $G_\delta x_1^{\beta} = G_\delta x_2^{\beta}$. An equivalent condition is to check whether $x_1^{\beta} (x_2^{\beta})^{-1} \in G_\delta$, which is then equivalent to $(x_1 x_2^{-1})^\beta \in G_\delta$. This is true as we have, by assumption $x_1 x_2^{-1} \in G_\omega$ and we know that $G_\omega = (G_\delta)^\alpha$ and therefore $G_\delta = (G_\omega)^\beta$.

Injectivity is proven similarly. Suppose $G_\omega x_1 \neq G_\omega x_2$. By the same argument we see that then also $\phi(G_\omega x_1) \neq \phi(G_\omega x_2)$.

Surjectivity is quick as well. Assume $G_\delta h \in G/G_\delta$. Clearly it holds that $\phi(G_\omega h^{\alpha}) = G_\delta h$ (remember, $\alpha^{-1} = \beta$).

So $\phi$ is a bijection. What remains to be proven is that $(\beta, \phi)$ is an isomorphism of actions. Since $\beta$ is by assumption a group automorphism and $\phi$ was just proven to be a bijection we must only check the following.
For every $g \in G$ and $G_\omega x \in G/G_\omega$ we must have:
\begin{equation*}
\phi(G_\omega x g) = \phi(G_\omega x) g,
\end{equation*}
plugging in our definition of $\phi$ we get that the above is equivalent to
\begin{equation*}
G_\delta (xg)^\beta = G_\delta x^\beta g^\beta,
\end{equation*}
which is true, since $\beta$ is an automorphism.

\end{document}
